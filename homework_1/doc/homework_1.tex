\documentclass[a4paper]{article}

\usepackage{amsmath, amssymb}
\usepackage{listings}
\usepackage{courier}
\usepackage{color}
\usepackage{pgfplots}
\usepackage{subcaption}


\pgfplotsset{compat=newest}
\usetikzlibrary{plotmarks}
\usetikzlibrary{arrows.meta}
\usepgfplotslibrary{patchplots}

\definecolor{light-gray}{gray}{0.7}

\newcommand{\norm}[1]{\left\lVert#1\right\rVert}

\author{H.T. Stoppels (University of Groningen)}
\title{{\sc nbalss} homework 1}

\lstdefinestyle{myCustomMatlabStyle}{
  language=Matlab,
  tabsize=4,
  showspaces=false,
  showstringspaces=false
}

\begin{document}
  \maketitle
  
  \paragraph{1} The model problem
  \begin{equation*}
  \begin{aligned}
    u^{\prime\prime}(x) &= \exp(x) \text{ on } (0, 1) \\
    u^\prime(1) &= e \\
    u(0) &= 1.
  \end{aligned}
  \end{equation*}
  clearly has the solution $u(x) = \exp(x).$

  \paragraph{2} Let $N$ be the number of segments, define the mesh-width $h := 1/N$ and the grid points $x_i := ih$ for $i = 0, 1, \dots, N.$ Let $u_i$ be the finite difference approximation to $u(x_i).$ Central difference discretization yields the approximation
  \begin{equation}\label{eq:local_approx}
    u^{\prime\prime}(x_i) = \frac{u_{i - 1} - 2u_i + u_{i+1}}{h^2} + O(h^2) \text{ as } h \to 0
  \end{equation}
  so that for the interior grid points we require
  \begin{equation}\label{eq:finite-diff-interior}
    u_{i-1} - 2u_i + u_{i+1} = h^2 \exp(x_i) \quad \text{ for } i = 2, \dots, N - 1.
  \end{equation}
  The Dirichlet boundary condition can of course be substituted
  \begin{equation}\label{eq:finite-diff-dirichlet}
  \begin{aligned}
    u_0 &= 1 \\
    -2u_1 + u_2 &= h^2 \exp(x_1) - 1.
  \end{aligned}
  \end{equation}
  The discretized Neumann boundary condition reads $$\frac{u_{N+1} - u_{N-1}}{2h} = e$$ so that the off-grid node $u_{N+1} := u_{N-1} + 2he$ can be substituted in \eqref{eq:finite-diff-interior} for $i = N,$ which gives the last equation
  \begin{equation}\label{eq:finite-diff-neumann}
    2u_{N - 1} - 2u_N = h^2\exp(x_N) - 2he.
  \end{equation}

  \paragraph{3} Combining~\eqref{eq:finite-diff-interior},~\eqref{eq:finite-diff-dirichlet} and~\eqref{eq:finite-diff-neumann} in matrix form
  \begin{equation*}
    \begin{bmatrix}
      1 \\
      & -2 & 1 &   &   &    \\
      &1 & -2 & 1 &   &    \\
      & & \ddots & \ddots & \ddots \\
      &  & & 1 & -2 & 1 &    \\
      & & & & 2 & -2
    \end{bmatrix}
    \begin{bmatrix}
      x_0 \\ x_1 \\ x_2 \\ \vdots \\ x_{N - 1} \\ x_N
    \end{bmatrix}
    =
    \begin{bmatrix}
      1 \\ h^2 \exp(x_1) - 1 \\ h^2 \exp(x_2) \\ \vdots \\ h^2 \exp(x_{N - 1}) \\ h^2 \exp(x_N) - 2he
    \end{bmatrix}
  \end{equation*}
  In practice one would omit the dummy unknown $x_0.$

  \paragraph{4} Using the following blueprint for central differences of a second derivative

  \lstset{basicstyle=\footnotesize\ttfamily,breaklines=true,style=myCustomMatlabStyle}
  \lstset{frame=tblr,rulecolor=\color{light-gray}}

  \begin{lstlisting}
function A = central_diff(n)
  e = ones(n, 1);
  A = spdiags([e, -2 * e, e], -1 : 1, n, n);
end
  \end{lstlisting}
  Applied to this specific problem
  \begin{lstlisting}
function [A, b, x_grid] = discretize(rhs, n)
  % Discretizes the problem using n rather than n + 1
  % grid points, omitting the dirichlet grid point at x_0.

  h = 1 / n;
  x_grid = linspace(0, 1, n + 1)';
  ys = rhs(x_grid);

  % Central difference matrix A + r.h.s. b
  A = central_diff(n);
  b = h * h * ys(2 : end);
  
  % Dirichlet left
  b(1) = b(1) - 1;

  % Neumann right
  b(end) = b(end) - 2 * h * exp(1);
  A(n, n - 1) = 2;
end
  \end{lstlisting}
  With these tools one can write
  \begin{lstlisting}
function ex1_4
  % Exact solution
  exact = @(x) exp(x);

  % Solve for one value of N to plot the error
  [A, b, x_grid] = discretize(@(x) exp(x), 32);
  
  % Solve Ax = b and prepend the Dirichlet grid point.
  numerical = [1; A \ b];

  err = numerical - exact(x_grid);

  figure;
  plot(x_grid, abs(err));
  title('Local error')
  xlabel('x')
  ylabel('error')

  fprintf('Global error (inf norm) = %f\n', norm(err, Inf));
end
  \end{lstlisting}
  Which outputs {\tt Global error (inf norm) = 0.000361} for the global error and produces the plot of {\bf Figure~\ref{fig:ex4}}.

  \paragraph{5} To show that central differences is a second-order approximation one should note that equation~\eqref{eq:local_approx} shows a \emph{local} second-order error in $h,$ however, here we show empirically that it has second-order convergence globally as well---in the sense that $\norm{u - u_h}_\infty = O(h^2)$ as $h \to 0$ where $u$ is the exact solution in the grid points and $u_h$ the finite differences approximation.

  Using the functions of above, we produce a plot for various values of $N$ via the following code
  \begin{lstlisting}
function ex1_5
  es = [];
  ns = 2 .^ (3 : 11);
  exact = @(x) exp(x);

  % Show experimentally 2nd-order convergence
  for n = ns
    [A, b, x_grid] = discretize(@(x) exp(x), n);

    % Solve Ax = b and prepend the Dirichlet grid point.
    numerical = [1; A \ b];

    % Compute the normed error (max norm)
    es(end + 1) = norm(numerical - exact(x_grid), Inf);
  end

  figure;
  loglog(ns, es, 'b'); hold on;
  loglog(ns, ns .^ -2, '-.'); hold off;
  grid
  title('Convergence rate central differences');
  xlabel('N');
  ylabel('Global error in inf norm');
  legend('Error', '1 / N^2')
end
  \end{lstlisting}
  which produces the plot seen in Figure~\ref{fig:ex5}, where one clearly sees the global error is second order in $h$ as well.

  \begin{figure}[b]
    \centering
    \begin{subfigure}[c]{0.47\textwidth}
      \caption{Local error for $N = 32$}
      \definecolor{mycolor1}{rgb}{0.00000,0.44700,0.74100}%
\begin{tikzpicture}

\begin{axis}[%
width=0.8\textwidth,
height=0.6*\textwidth,
at={(0.772in,0.516in)},
scale only axis,
xmin=0,
xmax=1,
xlabel style={font=\color{white!15!black}},
xlabel={$x$},
ymin=0,
ymax=0.0004,
% ylabel style={font=\color{white!15!black}},
% ylabel={$|u(x_i) - u_i|$},
axis background/.style={fill=white},
legend style={at={(0.01,0.99)}, anchor=north west, legend cell align=left, align=left, draw=white!15!black}
]
\addplot [color=mycolor1, mark=*, mark options={solid}, mark size=0.85]
  table[row sep=crcr]{%
0	0\\
0.03125	9.49599464417084e-06\\
0.0625	1.90739875567392e-05\\
0.09375	2.87365816415974e-05\\
0.125	3.84864624287662e-05\\
0.15625	4.83264006954087e-05\\
0.1875	5.82592551732208e-05\\
0.21875	6.82879753390875e-05\\
0.25	7.84156042961115e-05\\
0.28125	8.86452817470129e-05\\
0.3125	9.89802470581225e-05\\
0.34375	0.000109423842425516\\
0.375	0.000119979516136848\\
0.40625	0.000130650825941325\\
0.4375	0.000141441442522039\\
0.46875	0.000152355153082651\\
0.5	0.000163395865045546\\
0.53125	0.000174567609867893\\
0.5625	0.000185874546979159\\
0.59375	0.000197320967843417\\
0.625	0.000208911300149772\\
0.65625	0.000220650112138454\\
0.6875	0.00023254211705992\\
0.71875	0.000244592177779612\\
0.75	0.000256805311528385\\
0.78125	0.000269186694799028\\
0.8125	0.000281741668405111\\
0.84375	0.000294475742695255\\
0.875	0.000307394602935496\\
0.90625	0.000320504114860398\\
0.9375	0.000333810330405804\\
0.96875	0.000347319493613885\\
1	0.000361038046735818\\
};
\addlegendentry{$|u(x_i) - u_i|$}
\end{axis}
\end{tikzpicture}%
      \label{fig:ex4}
    \end{subfigure}
    \begin{subfigure}[c]{0.47\textwidth}
      \caption{Convergence rate}
      % This file was created by matlab2tikz.
%
%The latest updates can be retrieved from
%  http://www.mathworks.com/matlabcentral/fileexchange/22022-matlab2tikz-matlab2tikz
%where you can also make suggestions and rate matlab2tikz.
%
\definecolor{mycolor1}{rgb}{0.00000,0.44700,0.74100}%
%
\begin{tikzpicture}

\begin{axis}[%
width=0.8\textwidth,
height=0.6*\textwidth,
at={(0.758in,0.481in)},
scale only axis,
xmode=log,
xmin=1,
xmax=10000,
xminorticks=true,
xlabel style={font=\color{white!15!black}},
xlabel={$N$},
ymode=log,
ymin=1e-08,
ymax=0.1,
yminorticks=true,
% ylabel style={font=\color{white!15!black}},
% ylabel={$\norm{u - u_h}_\infty$},
axis background/.style={fill=white},
xmajorgrids,
xminorgrids,
ymajorgrids,
yminorgrids,
legend style={legend cell align=left, align=left, draw=white!15!black}
]
\addplot [color=blue, mark=*, mark options={solid}, mark size=0.85]
  table[row sep=crcr]{%
8 0.00577410736781836\\
16  0.00144402706770697\\
32  0.000361038046735818\\
64  9.0261466867414e-05\\
128 2.25654890164684e-05\\
256 5.6413795559429e-06\\
512 1.41034716349964e-06\\
1024  3.52584069052142e-07\\
2048  8.813638574523e-08\\
};
\addlegendentry{$\norm{u - u_h}_\infty$}

\addplot [color=mycolor1, dashdotted]
  table[row sep=crcr]{%
8 0.015625\\
16  0.00390625\\
32  0.0009765625\\
64  0.000244140625\\
128 6.103515625e-05\\
256 1.52587890625e-05\\
512 3.814697265625e-06\\
1024  9.5367431640625e-07\\
2048  2.38418579101562e-07\\
};
\addlegendentry{$1 / N^2$}

\end{axis}
\end{tikzpicture}%
      \label{fig:ex5}
    \end{subfigure}
  \end{figure}
\end{document}