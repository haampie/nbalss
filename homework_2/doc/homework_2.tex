\documentclass[a4paper]{article}

\usepackage{amsmath, amssymb}
\usepackage{listings}
\usepackage{courier}
\usepackage{color}
\usepackage{pgfplots}
\usepackage{subcaption}


\pgfplotsset{compat=newest}
\usetikzlibrary{plotmarks}
\usetikzlibrary{arrows.meta}
\usepgfplotslibrary{patchplots}

\definecolor{light-gray}{gray}{0.7}

\newcommand{\norm}[1]{\left\lVert#1\right\rVert}
\DeclareMathOperator{\diag}{diag}

\author{H.T. Stoppels (University of Groningen)}
\title{{\sc nbalss} homework 2}

\lstdefinestyle{myCustomMatlabStyle}{
  language=Matlab,
  tabsize=4,
  showspaces=false,
  showstringspaces=false
}

\lstset{basicstyle=\footnotesize\ttfamily,breaklines=true,style=myCustomMatlabStyle}
\lstset{frame=tblr,rulecolor=\color{light-gray}}

\begin{document}
  \maketitle
  
  \noindent The buckling problem for $\phi = \phi(s)$ is the non-linear ODE
  \begin{equation}\label{eq:problem}
  \begin{aligned}
    \phi^{\prime\prime} + \mu \sin \phi &= 0 \text{ on } (0, 1) \\
    \phi(0) &= 0 \\
    \phi(1) &= 0.
  \end{aligned}
  \end{equation}

  \paragraph{1} For any $\mu \in \mathbb{R}$ it is clear that the trivial solution $\phi \equiv 0$ satisfies~\eqref{eq:problem}.

  \paragraph{2} Using the Taylor expansion $\sin x = x + O(x^2)$ around $x = 0$ we obtain the linearized version of~\eqref{eq:problem} as $$\phi^{\prime\prime} = - \mu \phi$$
  which has the solutions $\phi(s) = \sin(\sqrt{\mu}s)$ satisfying the boundary condition at $s = 1$ if and only if $\mu = n^2\pi^2$ for $n \in \mathbb{N}.$ These solutions to the linearized equations are of relevance to the non-linear equation as can be seen from the following formulation of the problem: let $u(s) := \phi^\prime(s)$ and $v(s) := \phi(s)$ so that~\eqref{eq:problem} is equivalent to
  \begin{equation}
    \frac{d}{ds} \begin{bmatrix}
      u \\ v
    \end{bmatrix}
    =
    \begin{bmatrix}
      -\mu \sin v \\
      u
    \end{bmatrix}
  \end{equation}
  for which $u \equiv v \equiv 0$ is an equilibrium point. Its Jacobian matrix at this equilibrium is
  \begin{equation}
    \begin{bmatrix}
      0 & -\mu \\
      1 & 0
    \end{bmatrix}
  \end{equation}
  which is hyperbolic for the $\mu$'s we consider, so that the Hartman--Grobman theorem applies. This theorem formally states that the linearization provides qualitative information about the non-linear system.

  \paragraph{3} Let $n + 1$ be the number of segments, $h := 1 / (n + 1)$ the mesh-width and $\phi^h_j$ an approximation to $\phi(x_j)$ for $x_j = jh$ and $j = 0, \dots, n + 1.$ The boundary conditions give $\phi^h_0 = \phi^h_{n+1} = 0,$ so we exclude them from the computational domain. Define the matrix and vector
  \begin{equation}
    A = \begin{bmatrix}
      -2 & 1  &   &        & \\
      1  & -2 & 1 &  & \\
         & \ddots & \ddots & \ddots \\
         &        & 1      & -2 & 1\\
         &        &        & 1  & -2
    \end{bmatrix} \text{ and }
    \phi^h = \begin{bmatrix}
      \phi^h_1 \\ \phi^h_2 \\ \vdots \\ \phi^h_{n-1} \\ \phi^h_n
    \end{bmatrix}
  \end{equation}
  Then we obtain the difference equations
  \begin{equation}\label{eq:nonlinear-discretized}
    f(\phi^h) := A\phi^h + h^2\mu \sin \phi^h = 0
  \end{equation}
  for the non-linear case where $\sin \phi^h$ is understood element-wise, and
  \begin{equation}\label{eq:linearized}
    (A + h^2\mu I)\phi^h = 0
  \end{equation}
  for the linearized case. The elements on the diagonal of~\eqref{eq:linearized} are given by $-2 + h^2\mu$ and its off-diagonal elements are just $1,$ so for all relevant $\mu = n^2\pi^2$ with $n\in \mathbb{N}$ the matrix $A + \mu I$ loses diagonal dominance.

  \paragraph{4} Solving~\eqref{eq:nonlinear-discretized} with Newton's method requires the Jacobian of $f(\phi^h)$ which is of course easily determined as $J(\phi^h) = A + \mu \diag \cos \phi^h$ where $\cos \phi^h$ is again element-wise, and $\diag$ maps a vector to a diagonal matrix of the corresponding values. This allows us to write the following implementations:

  \begin{lstlisting}
function f = fBuck(theta, n, mu)
  A = linBuck(n);
  h_inv = (n + 1)^2;
  f = A * theta + mu * sin(theta) / h_inv;
end

function J = JacBuck(theta, n, mu)
  A = linBuck(n);
  h_inv = (n + 1)^2;
  J = A + spdiags(mu * cos(theta) / h_inv, 0, n, n);
end

function A = linBuck(n)
  e = ones(n, 1);
  A = spdiags([e, -2 * e, e], -1 : 1, n, n);
end
  \end{lstlisting}
  Next, it should hold that $$f(\phi^h + \varepsilon v) = f(\phi^h) + \varepsilon J(\phi)v + O(\varepsilon^2) \text{ as } \varepsilon \to 0$$ in any norm on $\mathbb{R}^n$, for $\phi^h$ fixed, a unit vector $v \in \mathbb{R}^n$ and $0 < \varepsilon \ll 1.$ As a sanity check we generate $v$ as an element-wise uniformly randomly distributed (between $-0.5$ and $0.5$) vector which is then normalized in the 2-norm. Next, we iterate $\varepsilon$ through $\{10^{-1}, 10^{-2}, \dots, 10^{-12}\}$ and measure the normed residual 
  \begin{equation}\label{eq:normed-res}
    r(\varepsilon) = \norm{f(\phi^h + \varepsilon v) - f(\phi^h) - \varepsilon J(\phi)v}_2.
  \end{equation} 
  In Figure~\ref{fig:normed-res} (on the last page) we plot $r(\varepsilon)$ for $\phi^h_j = \sin \pi x_j$ and $v$ as mentioned above for $n = 40$. The residual reduces quadratically as $\varepsilon \to 0,$ which is to be expected; yet is halted when $\varepsilon \le 10^{-6}.$ At this point the error in the matrix-vector products dominates the perturbation in $\phi^h.$ The code for generating this plot is the following:
  \newpage
\begin{lstlisting}
function ex2_4(n, mu)
  h = 1 / (n + 1);
  xs = linspace(h, 1 - h, n).';
  theta = sin(sqrt(mu) * xs);

  v = rand(n, 1) - 0.5;
  v = v / norm(v);

  epsilons = 10 .^ -(1 : 12);
  errors = [];
  J = JacBuck(theta, n, mu);

  for epsilon = epsilons
    func = fBuck(theta + epsilon * v, n, mu);
    approx = fBuck(theta, n, mu) + epsilon * J * v;
    errors(end + 1) = norm(func - approx, Inf);
  end

  figure;
  loglog(1 ./ epsilons, epsilons .^ 2, '-+'); hold on;
  loglog(1 ./ epsilons, errors, '-+'); hold off;
end
\end{lstlisting}
  And is called via
  \begin{lstlisting}
ex2_4(40, pi ^ 2)
  \end{lstlisting}

  \begin{figure}[b]
    \caption{Normed residual of equation~\eqref{eq:normed-res} for $n = 40.$ }
    \label{fig:normed-res}
    \definecolor{mycolor1}{rgb}{0.00000,0.44700,0.74100}%
\definecolor{mycolor2}{rgb}{0.85000,0.32500,0.09800}%

\begin{tikzpicture}

\begin{axis}[%
width=0.85\linewidth,
height=0.5\linewidth,
at={(0.772in,0.516in)},
scale only axis,
xmode=log,
xmin=1,
xmax=1000000000000,
xminorticks=true,
xlabel style={font=\color{white!15!black}},
xlabel={$\varepsilon^{-1}$},
ymode=log,
ymin=1e-25,
ymax=1,
yminorticks=true,
axis background/.style={fill=white},
xmajorgrids,
xminorgrids,
ymajorgrids,
yminorgrids,
legend style={legend cell align=left, align=left, draw=white!15!black}
]
\addplot [color=mycolor1, mark=+, mark options={solid, mycolor1}]
  table[row sep=crcr]{%
10	0.01\\
100	0.0001\\
1000	1e-06\\
10000	1e-08\\
100000	1e-10\\
1000000	1e-12\\
10000000	1e-14\\
100000000	1e-16\\
1000000000	1e-18\\
10000000000	1e-20\\
100000000000	1e-22\\
1000000000000	1e-24\\
};
\addlegendentry{$\varepsilon^2$}

\addplot [color=mycolor2, mark=+, mark options={solid, mycolor2}]
  table[row sep=crcr]{%
10	0.00605709447273181\\
100	6.05626647845522e-05\\
1000	6.05618465945497e-07\\
10000	6.05639314430789e-09\\
100000	6.05686991120942e-11\\
1000000	1.89571633311603e-12\\
10000000	1.08449017713603e-12\\
100000000	6.9539705802595e-13\\
1000000000	1.67883800867209e-12\\
10000000000	1.2269149419109e-12\\
100000000000	1.12686013860168e-12\\
1000000000000	1.40820104633013e-12\\
};
\addlegendentry{$r(\varepsilon)$}

\end{axis}
\end{tikzpicture}%
  \end{figure}

  \paragraph{5} Newton's method is easily implemented with the following snippet (without {\tt feval} and explicit parameters):
\begin{lstlisting}
function x = newton(f, df, x, tol, varargin)
  incr = inf;

  while norm(incr, 2) > tol
    incr = df(x, varargin{:}) \ f(x, varargin{:});
    x = x - incr;
  end
end
\end{lstlisting}
  As an example, solving $x^2 -1 = 0$ for $x$ can be done as follows:
\begin{lstlisting}
  newton(@(x) x^2 - 1, @(x) 2 * x, 0.1, 1e-4)
\end{lstlisting}
  which returns {\rm 1.000000000013990} in a handful of iterations.
  \paragraph{6} A program setup that computes the zero of $f(\theta^h)$ for some initial guess of the form $\hat{\theta}^h_j = \sin \sqrt{\mu} x_j$ is along the following lines
\begin{lstlisting}
function theta_zero = zerofBuck(n, mu, tol)
  h = 1 / (n + 1);
  xs = linspace(h, 1 - h, n).';
  theta_start = sin(sqrt(mu) * xs);
  theta_zero = newton(@fBuck, @JacBuck, theta_start, tol, n, mu);

  % figure
  % plot(xs, theta_zero); hold on;
  % plot(xs, theta_start); hold off;
  % legend('Approximate solution', 'Initial guess')
end
\end{lstlisting}
\end{document}