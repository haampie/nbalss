%!TEX root = ../report.tex
\section{Introduction}
We consider a simplified, homogeneous, two-dimensional, wind-driven ocean model in a square basin. After non-dimensionalizing the equations and assuming a constant density $\rho$ together with incompressibility, we can formulate the problem in terms of the stream function $\psi = \psi(t, x, y).$ This formulation basically eliminates the pressure term from the Navier-Stokes equations and automatically fulfills the continuity equation. [Something something boundary conditions.] Ultimately this leads to the following PDE:
\begin{equation}\label{eq:pde}
\begin{aligned}
  \left(\frac{\partial}{\partial t} + u \frac{\partial}{\partial x} + v \frac{\partial}{\partial y}\right) \zeta + \beta \frac{\partial \psi}{\partial x} &= -\alpha \frac{\partial \tau_x}{\partial y} + \frac{1}{Re} \Delta \zeta & \text{in } [0, \infty) \times [0,1]^2, \\
  \zeta &= \Delta \psi & \text{in } [0, \infty) \times [0,1]^2, \\
  \psi = v &= 0 & \text{on } x = 0 \text{ and } x = 1, \\
  \psi = \zeta &= 0 & \text{on } y = 0 \text{ and } y = 1
\end{aligned}
\end{equation}
where $u = -\frac{\partial \psi}{\partial y}$ and $v = \frac{\partial \psi}{\partial x}$ and the wind-stress forcing is
\begin{equation}
  \tau_x(t, x, y) = \frac{- \eta}{2 \pi} \cos{2\pi y}.
 \label{eq:tau}
\end{equation}
We keep $\alpha = \beta = 1000$ fixed, and start with $Re = 16$ and $\eta = 0,$ such that the trivial solution $\psi = 0$ satisfies~\eqref{eq:pde}. Furthermore, we discretize~\eqref{eq:pde} in space using a second-order central differences scheme on a uniform $128\times128$ grid so that the PDE reduces to a system of ODEs of the form
\begin{equation}\label{eq:ode}
  M \frac{d \*u}{d t} = F(\*u, p)
\end{equation}
with the unknown $\*u \in \mathbb{R}^{2n^2}$ and $M$ the mass matrix. Here $p$ can be interpreted as a single parameter we wish to modify, for instance $Re$ or $\eta.$ The total number of unknowns is $N = 2 n^2,$ since our grid is uniform and each grid point gives an equation for both $\psi$ and $\zeta.$

\subsection{Pseudo-arclength continuation theory}
We're looking for equilibria of~\eqref{eq:ode} where $F(\*u, p) = 0.$ In particular we want to follow an equilibrium as $p$ varies. To do so, we apply the following procedure:
\begin{enumerate}
  \item Parametrize $\Gamma(s) = (\*u(s), p(s))$ with $\|\dot\Gamma(s)\| = 1.$
  \item Find a starting point $\Gamma(s_0)$, which is not a bifurcation point, such that $F(\Gamma(s_0)) = 0.$ 
  \item Compute $\dot\Gamma(s_0)$ explicitly (only once).
  \item Repeat for $k = 0, 1, \dots$ the predictor-corrector iteration for fixed $\Delta s$ until you reach a point of interest.
  \begin{description}
    \item[Predict:] Fix $\Delta s.$ Compute the guess $$\Gamma^{(k+1)} := \Gamma(s_k) + \Delta s * \dot\Gamma(s_k).$$
    \item[Correct:] Set $s_{k+1} = s_k + \Delta s$ and solve the non-linear equations \begin{equation}\begin{aligned}\label{eq:hoi}
      F(\Gamma(s_{k+1})) &= 0, \\
      \dot\Gamma(s_k)^T\Gamma(s_{k+1}) &= \Delta s + \dot\Gamma(s_k)^T\Gamma(s_k).
    \end{aligned}
    \end{equation}
    where $\Gamma(s_{k+1})$ is the only unknown, via a Newton-Rhapson procedure with $\Gamma^{(k+1)}$ the initial guess.
  \end{description}
\end{enumerate}
In our case we have in step 2 the trivial solution at our disposal, but this requires that $\eta = 0$ initially. Step 3 is more involved: since $F(\Gamma(s)) = 0$ for all $s$ per definition, we differentiate with respect to $s$ and using the chain rule we find after evaluation in $s_0$ that
\begin{equation}\label{eqn:null}
  \begin{bmatrix}
    DF_u(s_0) & DF_p(s_0)
  \end{bmatrix}\Dot\Gamma(s_0)
  = 0 \quad \text{ and } \quad \|\dot\Gamma(s_0)\| = 1
\end{equation}
where $DF_z$ denotes the Jacobion matrix of $F$ with respect to $z.$ Equation~\eqref{eqn:null} means $\dot\Gamma(s_0)$ is a unit vector in the null space of the $N \times (N + 1)$ dimensional matrix $$D := \begin{bmatrix} DF_u(s_0) & DF_p(s_0) \end{bmatrix}.$$ Under the assumption that $\Gamma(s_0)$ is not a bifurcation point, this matrix $D$ has full rank, so the null space is indeed one-dimensional and hence \eqref{eqn:null} has a unique solution. Computing the null space can be performed by bringing $D$ into row-echelon form as $\tilde{D}$ and solving the $(N + 1) \times (N + 1)$ augmented linear system of equations
\begin{equation}
  \begin{bmatrix}
    \tilde{D} \\ e_{N+1}^T
  \end{bmatrix}
  \dot\Gamma(s_0)
  =
  e_{N}
\end{equation}
where $e_{i}$ is the $i$th standard basis vector of length $i.$ 

Lastly, we refer to the the lecture notes for the technical details of step 4, yet we will show how Equation~\eqref{eq:hoi} is obtained. Rather than requiring exactly $\|\dot\Gamma(s)\| = 1$ in the correction phase, we require only
\begin{equation}
  \|\dot\Gamma(s_k)\| \approx \dot\Gamma(s_k)^T \frac{\Gamma(s_{k+1}) - \Gamma(s_k)}{\Delta s} = 1
\end{equation}
which explains the wording {\em pseudo}-arclength. After rewriting this, it can be seen to be equivalent to the second equation of~\eqref{eq:hoi}.