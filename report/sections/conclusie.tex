\section{Conclusion}
We have looked at the steady states of the homogeneous wind-driven ocean circulation. We started with a trivial steady state with no wind-stress forcing and Reynolds number 16. From this steady state, we used a pseudo-arclength continuation in the wind forcing till we had full wind forcing. We found a steady state for which the streamfunction was symmetric. From this steady state, we applied a continuation in the Reynolds number to find a pitchfork bifurcation. This bifurcation appeared to be at $Re_p=29.5112$. In order to find the branches of asymmetric solutions, we started at the steady state for $Re=16$. We broke the symmetry of the pitchfork by adding a non-symmetric forcing term. We applied a continuation in Re in this asymmetric system till we passed the bifurcation point. When we where at $Re\approx 31$, we went back to the original symmetric forcing system. So we and up on one of the branches of asymmetric solution. We used again the pseudo-arclength continuation, but now backwards in Re, following the branch through the bifurcation point and we found the other asymmetric branch.

After we found the pitchfork bifurcation, we also tried to find a Hopf-bifurcation. We started at the upper asymmetric branch of the pitchfork. The Hopf-bifurcation can only be detected by looking at the eigenvalues. So we applied a continuation in Re again, but at each step, we computed the eigenvalues and found that at $Re_H=68.9781$ a conjugate eigenpair crosses the imaginary axis and thus is there a Hopf bifurcation.