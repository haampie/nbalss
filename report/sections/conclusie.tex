%!TEX root = ../report.tex
\section{Conclusion}
We look at the steady states of the homogeneous wind-driven ocean circulation, starting with a trivial steady state with no wind-stress forcing at Reynolds number $16$. From this steady state, we use pseudo-arclength continuation in the wind forcing untill we have full capacity. We find a steady state for which the streamfunction has a simple anti-symmetric form around $y = 0.5$. From this steady state, we apply continuation in the Reynolds number to find a pitchfork bifurcation. This bifurcation appears at $Re_p \approx 29.5112$. In order to find the upper and lower branches of equilibria, we break the (anti-)symmetry of the equations by adding a non-symmetric forcing term. Then we apply continuation in $Re$ in this asymmetric system past the bifurcation point at $Re\approx 31$, where we restore symmetry again, so we end up on the upper branch. We then use pseudo-arclength continuation again, but now backwards in $Re$, following the branch through the bifurcation point and reaching the lower counterpart of the pitchfork.

After finding the pitchfork bifurcation, we also look for a Hopf-bifurcation. Continuing at the upper branch of the pitchfork, the Hopf-bifurcation can only be detected by looking at the eigenvalues. So we apply continuation in $Re$ again, but at each step, we compute the eigenvalues and find that at $Re_H \approx 68.9781$ a conjugate eigenpair crosses the imaginary axis, which indicates the location of the Hopf bifurcation.