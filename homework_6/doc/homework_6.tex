\documentclass[a4paper]{article}

\usepackage{amsmath, amssymb}
\usepackage{algorithm}
\usepackage{algpseudocode}
\usepackage{listings}
\usepackage{color}
\usepackage{pgfplots}
\usepackage[parfill]{parskip}
\usepackage{subcaption}
\usepackage[colorlinks = true, urlcolor=red]{hyperref}
\usepackage[font={small,it}]{caption}

\pgfplotsset{compat=newest}
\usetikzlibrary{plotmarks}
\usetikzlibrary{arrows.meta}
\usepgfplotslibrary{patchplots}

\newcommand{\norm}[1]{\left\lVert#1\right\rVert}
\DeclareMathOperator{\diag}{diag}
\DeclareMathOperator{\sspan}{span}

\author{H.T. Stoppels (University of Groningen)}
\title{{\sc nbalss} Homework 6}

\begin{document}
  \maketitle 

  \paragraph{1} The \emph{Orthogonal Iteration Method} is a generalization of the Power Method for finding $k$ eigenpairs of an $n \times n$ matrix $A$ corresponding to the eigenvalues of largest absolute magnitude.

  \begin{algorithm}
\caption{Orthogonal Iteration Method}\label{alg:oim}
\begin{algorithmic}[1]
\Procedure{OIM}{$A$}
\State $V^{(1)}$ random $n \times k$ orthonormal matrix.
\For{$i = 1, 2, \dots$ {\bf until convergence}}
  \State $Z^{(i+1)} = A V^{(i)}$
  \State Compute eigenvalues of $(V^{(i)})^*Z^{(i+1)}$
  \State $V^{(i+1)}R^{(i+1)} = Z^{(i+1)}$  \Comment{QR decomposition}
\EndFor
\EndProcedure
\end{algorithmic}
\end{algorithm}
  To see how it works, note that the ordinary power method is applied to the first column $v_1^{(1)}$ of $V^{(1)},$ since it is only normalized and not orthogonalized. The power method is applied to $v_2^{(1)}$ as well, with the only difference that it is not only normalized at each iteration $i$, but orthogonalized w.r.t. $v_1^{(i)}$ as well. Hence as $v_1^{(i)}$ converges to the eigenvector corresponding to the eigenvalue of largest absolute magnitude, we see in the second column that we apply the power method on a vector restricted to the subspace $\mathbb{C}^n \setminus \sspan\{v_1^{(i)}\},$ from which we conclude that the power method can only convergence to the eigenvector corresponding to the second largest eigenvalue in absolute magnitude. The same holds for the remaining columns.

  Next, we can also apply OIM to $A^{-1},$ which will converge to eigenvector corresponding to the smallest eigenvalues of $A,$ because $Ax = \lambda x$ iff $A^{-1}x = \lambda^{-1}x$ and $\lambda^{-1}$ is largest when $\lambda$ is smallest in absolute sense. Note that the eigenvectors are identical. In practice we factorize $A = LU$ (or Cholesky) and perform multiplication $y = A^{-1}z$ by solving $Lx = z$ followed by $Uy = x.$

  In Julia, I have chosen to add a compile-time option to switch between inverted and ordinary orthogonal iterations. Passing a factorized matrix object and choosing {\tt Invert()} will perform the matrix-vector operation with the backslash operator, which is overridden by the factorized matrix object to efficiently apply $A^{-1}$ to another matrix or vector. Lastly, this modus automatically inverts the eigenvalues, so there is no need to do it manually afterwards.
  \begin{lstlisting}
A = poisson(200)
F = factorize(A) # F is a factorized matrix object
V, D, = orthogonal_iteration(F, 3, Invert()) # Inverted iteration
  \end{lstlisting}

  In Figure~\ref{fig:lambda_conv} we see the convergence history of the eigenvalues which is defined simply as $|\lambda^{(i + 1)}_k - \lambda^{(i)}_k|$ for $k = 1, 2, 3.$ Clearly, the farther away the eigenvalue is from the origin, the slower it converges, which is in line with the convergence behaviour of the inverted power method: the fraction $|\lambda_k| / |\lambda_{k+1}|$ determines the rate of convergence for the $k$th eigenvalue.

  \begin{figure}[h]
    \caption{Convergence history $|\lambda^{(i + 1)}_k - \lambda^{(i)}_k|$ of the smallest eigenvalues $\lambda_1, \lambda_2, \lambda_3$ of the Poisson matrix}\label{fig:lambda_conv}
    \centerline{\includegraphics{images/convergence.pdf}}
  \end{figure}

  \begin{itemize}
    \item For the implementation of the OIM see \url{https://github.com/haampie/nbalss/blob/master/homework_6/NBALSS.jl/src/orthogonal_iteration.jl#L17-L43}
    \item The plot of Figure~\ref{alg:oim} was generated with \url{https://github.com/haampie/nbalss/blob/master/homework_6/NBALSS.jl/src/exercises.jl#L14-L30}
    \item The {\tt init} procedure was updated in \url{https://github.com/haampie/nbalss/blob/master/homework_6/NBALSS.jl/src/init.jl#L4-L6}
  \end{itemize}

  \paragraph{2} Assuming we are interested in the stability of the \emph{spatially discretized} problem only --- which is a system of non-linear ODEs $\theta_t(s) = f(\theta, s)$ --- and assuming we know an equilibrium solution $\theta^*$ s.t. $f(\theta^*, s) = 0,$ we must have that the real part of the eigenvalues of the Jacobian $$J(\theta^*) = \frac{\partial f}{\partial \theta}(\theta^*, s) = A + \mu \diag \cos \theta^*$$ have strictly negative real part. (Here A is the discretized differential operator, $\cos$ is taken element-wise and $\diag$ maps a vector to a diagonal matrix.)

  So for a given $\theta^*$ we should find the eigenvalues of $J(\theta^*)$ which have passed the imaginary axis. Since the eigenvalues are known to be real, and assuming we are close to a bifurcation point, we can simply search for a few eigenvalues close to the origin with the method of exercise {\bf 1}.

  \paragraph{3} To investigate the stability of the branches I considered the problem $$\theta_t = \theta_{xx} + \sin \theta - 0.01 s(1-s)$$ with the familiar boundary conditions, discretized in space with $n = 300$ grid points.
  \begin{enumerate}
    \item I wrote a custom function {\tt continuation\_with\_stability\_check} which performs continuation in $\mu,$ but at each step computes the 3 eigenvalues of the Jacobian closest to the origin. If there is any positive eigenvalue it prints {\tt unstable} and otherwise {\tt stable}.
    \item First I took $\theta = 0$ at $\mu = 30$ and applied Newton to get on the nearly trivial branch. Then I performed (backward) continuation from $\mu = 30$ all the way back to $\mu = 0.$ The only notable change happens when the branch switches from {\tt unstable} to {\tt stable} when $\mu$ steps from 9.87 to 9.84; which is indeed the first bifurcation point.
    \item Then I performed continuation from $\mu = 0$ to $\mu = 30,$ which automatically gets on the non-trivial branch. This branch happens to be {\tt stable} everywhere.
    \item Lastly I switched signs $\theta \gets -\theta$ at $\mu = 30,$ applied Newton to get on the other non-trivial branch and performed continuation from $\mu = 30$ back to $\mu = 0$ again. Also nothing special happens, this branch is stable as well.
  \end{enumerate}
  So to recap, the nearly trivial branch is the only unstable branch in the interval of interest $\mu \in [20, 30],$ which is to be expected from an imperfect pitchfork bifurcation.

  The code is available here:
  \begin{itemize}
    \item Implementation of the stability check in continuation: \url{https://github.com/haampie/nbalss/blob/master/homework_6/NBALSS.jl/src/continuation.jl#L9-L24}
    \item And its application to check the stability of all branches: \url{https://github.com/haampie/nbalss/blob/master/homework_6/NBALSS.jl/src/exercises.jl#L52-L79}
  \end{itemize}

  \paragraph{4} Unfortunately I didn't find time to finish this exercise because of an exam.
\end{document}